\documentclass[a4paper,fleqn,12pt]{article}
%\usepackage[T1]{fontenc}
\usepackage[brazilian]{babel}
%\usepackage[utf8]{inputenc}
\usepackage[left=2.5cm,right=2.5cm,top=3cm,bottom=2.5cm]{geometry}
\usepackage{mathtools}
%\usepackage{amsthm}
%\usepackage{amsmath}
%\usepackage{nccmath}
%\usepackage{amssymb}
\usepackage{amsfonts}
\usepackage{physics}
%\usepackage{dsfont}
%\usepackage{mathrsfs}

\usepackage{titling}
\usepackage{indentfirst}

\usepackage{bm}
\usepackage[dvipsnames]{xcolor}
\usepackage{cancel}

\usepackage{xurl}
\usepackage[colorlinks=true]{hyperref}

% code
\definecolor{bg}{rgb}{0.90,0.90,0.90}
\usepackage{minted}


%\usepackage{float}
%\usepackage{graphicx}
%\usepackage{tikz}
%\usepackage{caption}
%\usepackage{subcaption}

%%%%%%%%%%%%%%%%%%%%%%%%%%%%%%%%%%%%%%%%%%%%%%%%%%%

\newcommand{\eps}{\epsilon}
\newcommand{\vphi}{\varphi}
\newcommand{\cte}{\text{cte}}

\newcommand{\N}{\mathbb{N}}
\newcommand{\Z}{\mathbb{Z}}
\newcommand{\Q}{\mathbb{Q}}
\newcommand{\R}{\mathbb{R}}
%\newcommand{\C}{\mathbb{C}}
\renewcommand{\H}{\hat{H}}
\newcommand{\intR}{\int_{-\infty}^{\infty}}

\newcommand{\0}{\vb{0}}
\newcommand{\1}{\mathds{1}}
\newcommand{\E}{\vb{E}}
\newcommand{\B}{\vb{B}}
\renewcommand{\v}{\vb{v}}
\renewcommand{\r}{\vb{r}}
\renewcommand{\k}{\vb{k}}
\newcommand{\p}{\vb{p}}
\newcommand{\q}{\vb{q}}
\newcommand{\F}{\vb{F}}

\renewcommand{\a}{\hat{a}}
\renewcommand{\b}{\hat{b}}
\renewcommand{\c}{\hat{c}}
\newcommand{\nn}{\hat{n}}

\newcommand{\gf}[2]{\ev{\ev{#1 : #2}}}
\newcommand{\zub}[2]{\ev{\comm{#1}{#2}_\mp}}

\newcommand{\s}[1]{\mathcal{#1}}
%\newcommand{\prodint}[2]{\left\langle #1 , #2 \right\rangle}
\newcommand{\cc}[1]{\overline{#1}}
\newcommand{\Eval}[3]{\eval{\left( #1 \right)}_{#2}^{#3}}

\newcommand{\unit}[1]{\; \mathrm{#1}}

\newcommand{\n}{\medskip}
\newcommand{\e}{\quad \mathrm{e} \quad}
\newcommand{\ou}{\quad \mathrm{ou} \quad}
\newcommand{\virg}{\, , \;}
\newcommand{\ptodo}{\forall \,}
\renewcommand{\implies}{\; \Rightarrow \;}
%\newcommand{\eqname}[1]{\tag*{#1}} % Tag equation with name

% math %
\renewcommand{\erf}[1]{\text{erf}\left(#1\right)}
\newcommand{\floor}[1]{\left\lfloor #1 \right\rfloor}
\newcommand{\ceil}[1]{\left\lceil #1 \right\rceil}

\setlength{\droptitle}{-6em}


\title{\Huge{\textbf{Exam 1, IA}}}
\author{Mateus Marques}

\usepackage[shortlabels]{enumitem}

\begin{document}

\maketitle

TODO: estudar muito os slides \texttt{evalution.pdf}.

\section{Question 1}
$E =$ classifier $A$ makes an error. $C =$ correct decision.
$$
\begin{cases}
\; P(E) = 1/4 = x \\
\; P(C | \neg E) = 9/10 = y \\
\; P(C | E) = 3/10 = z \\
\end{cases}
$$
We want $P(\neg E| C)$:
$$
P(\neg E | C) = \frac{P(C | \neg E) P(\neg E)}{P(C)} =
\frac{P(C|\neg E) (1-P(E))}{P(C|E) P(E) + P(C|\neg E) P(\neg E)} =
\frac{y (1-x)}{z x + y (1-x)}
$$
$$
= \frac{9/10 \cdot 3/4}{3/10 \cdot 1/4 + 9/10 \cdot 3/4} =
\frac{9 \cdot 3}{3 + 9 \cdot 3} = 9/10.
$$

\section{Question 2}

$$
P(X = x | Y = y) = \frac{5(1-y) + yx}{5 \cdot (y + 10)}.
$$
We know $P(Y=1) = 2/3$.
$$
P(Y = 1 | X = x) P(X = x) = P(X = x | Y=1) P(Y = 1).
$$
$$
P(Y = -1 | X = x) P(X = x) = P(X = x | Y=-1) P(Y = -1).
$$
Therefore:
$$
f(x) = \frac{P(Y=1|X=x)}{P(Y=-1|X=x)} = \frac{P(X=x|Y=1)}{P(X=x|Y=-1)} \, \frac{P(Y=1)}{P(Y=-1)} =
\frac{x/55}{5/50} \cdot 2 = \frac{20}{55} \, x.
$$
The Bayes classifier generates $Y = 1$ if $f(x) \geq 1$ and $Y = -1$ if $f(x) < 1$.

We have $f(2) = 40/45$ and $f(8) = 160/55$.

\n

The answer is $(-1, 1)$.

\section{Question 3}

$\hat{f}(X) = 2X$, but $Y = X+2$ and $p_X(x) = x/2$. Also $p_Y(y) \dd{y} = p_X(x) \dd{x}$.
$$
E[(Y - \hat{Y})^2] = \int_{y(0)}^{y(2)} (Y-\hat{Y})^2 \, p_Y(y) \dd{y} =
\int_{0}^{2} (Y(x)-\hat{Y}(x))^2 \, p_X(x) \dd{x} =
$$
$$
= \int_{0}^{2} (x + 2 - 2x)^2 \, x / 2 \dd{x} = \frac{1}{2} \int_0^2 (x^3-4x^2+4x) \dd{x} =
\frac{1}{2} \, \qty(\frac{16}{4} - 4 \cdot \frac{8}{3} + 8) = 2/3.
$$

\section{Question 4}

When analyzing only label $1$ or not, we have the following confusion matrix:
$$
\begin{pmatrix}
35 & 5 \\
14 & 80
\end{pmatrix}
$$
The precision is $\frac{\text{true positive}}{\text{get the label}} = \frac{35}{35 + 5} = 7/8$.

\section{Question 5}

\begin{enumerate}[(a)]
\item No. We only have a estimative of something $\eps_{\text{1NN}} \leq 2 \eps_{\text{Bayes}}$.
\item No. There are not a combination of classifiers.
\item That is correct. Simple.
\item No. We only have a estimative of something $\eps_{\text{1NN}} \leq 2 \eps_{\text{Bayes}}$.
\item Obviously false, kNN has nothing to do with a brain.
\end{enumerate}

\section{Question 6}

I don't know about this one.

\section{Question 7}

$$
\text{error rate} = \frac{2000}{10000} = 0.2.
$$
The $\sigma$ is, where $\eps(i) = 1$ if the classifier missed and $\eps(i) = 0$ if it got right.
$$
\sigma^2 = \frac{1}{10000} \sum_{i=1}^{10000} (0.2 - \eps(i))^2 =
\frac{2000 \cdot 0.8^2 + 8000 \cdot 0.2^2}{10000} = 0.16 \implies \sigma = 0.04.
$$
The confidence interval with confidence 0.95 corresponds to $2\sigma$, therefore $[0.2 - 2\sigma, 0.2 + 2\sigma] = [0.192, 0.28]$.

\textbf{THIS IS WRONG}, olhar nos slides.



\end{document}
